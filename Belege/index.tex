\documentclass[a4paper, 11pt]{article}

\usepackage[utf8]{inputenc}
\usepackage[ngerman]{babel}  % deutsche Silbentrennung
\usepackage[T1]{fontenc}
%\usepackage[round]{natbib}
\usepackage{graphicx}
\usepackage{geometry}
\usepackage{makeidx} % Package zur Indexerstellung
\usepackage{listings}
\usepackage{url}
\parindent 0pt
\usepackage{rotating}

\usepackage{graphicx} 
\usepackage{subfigure}

\geometry{outer=25mm , inner=25mm , bottom=25mm , top=25mm }	
\newcommand{\cmd}[1]{\textbackslash #1}
\newcommand{\meta}[1]{\textless#1\textgreater}
\newcommand{\val}[1]{\{\meta{#1}\}}
\newcommand{\valo}[1]{[\meta{#1}]}
\newcommand{\incsection}[1]{\include{kapitel/#1}}
\lstset{
basicstyle=\footnotesize\ttfamily, % Standardschrift
         numbers=left,               % Ort der Zeilennummern
         numberstyle=\tiny,          % Stil der Zeilennummern
         stepnumber=1,               % Abstand zwischen den Zeilennummern
         numbersep=5pt,              % Abstand der Nummern zum Text
         tabsize=2,                  % Groesse von Tabs
         extendedchars=true,         %
         breaklines=true,            % Zeilen werden Umgebrochen
         keywordstyle=\color{red},
         stringstyle=\color{white}\ttfamily, % Farbe der String
         showspaces=false,           % Leerzeichen anzeigen ?
         showtabs=false,             % Tabs anzeigen ?
         xleftmargin=17pt,
         framexleftmargin=17pt,
         framexrightmargin=5pt,
         framexbottommargin=4pt,
         %backgroundcolor=\color{lightgray},
         showstringspaces=false,      % Leerzeichen in Strings anzeigen ? 
         numberbychapter=true
}

\begin{document}
\setlength{\parindent}{0.0cm} % Erstzeilen-Einzug
\begin{titlepage}

\large
\begin{figure}
  \begin{center}
    \hbox to \hsize{%
      \begin{tabular}[m]{c}
        \includegraphics[width=4cm]{images/hochschule_zittau_goerlitz_logo.png}
      \end{tabular}
      \hfill%
      \begin{tabular}[m]{c}
        Hochschule Zittau/Görlitz\\
        Fakultät Elektrotechnik-Informatik \\
        
      \end{tabular}%
    }
  \end{center}
\end{figure}

\begin{center}
\rule{0pt}{0pt}

\begin{huge}
Belegarbeit Name des Faches
\end{huge}
\vspace{.5cm}
\begin{center}
bei\\
\vspace{.5cm}
Prof. so und so \\Hochschule Zittau/Görlitz
\end{center}


\vspace{2cm}

Kurze Erklärung des Themas.\\

\vspace*{2cm}

\begin{tabular}[m]{c}
Der Eine \\Hochschule Zittau/Görlitz\\ Matr.-Nr.: 12345
\end{tabular}
\vspace*{.5cm}

\begin{tabular}[m]{c}
Der Andere \\Hochschule Zittau/Görlitz\\ Matr.-Nr.: 67890
\end{tabular}

\vspace{3cm}

1. Februar 2013

\end{center}
\end{titlepage} % Referenzierung des Titelblattes
\tableofcontents  % Inhaltsverzeichnis anzeigen

\newpage
\thispagestyle{empty}
\addcontentsline{toc}{section}{Selbstständigkeitserklärung}

Hiermit erklären wir, dass wir diese Arbeit selbstständig verfasst und keine anderen als die angegebenen Quellen und Hilfsmittel benutzt haben.
Diese Arbeit wurde bisher auch keiner anderen Prüfungsbehörde vorgelegt und auch noch nicht veröffentlicht.

\vspace{5cm}

Görlitz, den 9.7.2014 %\hspace{4cm} \line(1,0){150}

\vspace{1cm}
\line(1,0){150}\\
Dominik Bitterlich

\vspace{1cm}
\line(1,0){150}\\
Alexander Häse

\vspace{1cm}
\line(1,0){150}\\
Daniel Richter

\vspace{1cm}
\line(1,0){150}\\
Tom Schumann
%Kapitel in orderner sections als eigene dateien
\incsection{Einleitung}

%content

% TODO anpassen, DIES HIER DIENT NUR ALS BEISPIEL
\addcontentsline{toc}{section}{Quellenverzeichnis}
\renewcommand{\refname}{Quellenverzeichnis}
\def\UrlFont{\bfseries}
\nocite{*}
\begin{thebibliography}{99}
	\bibitem{1} \url{https://developers.google.com/apps-script/articles/bigquery_tutorial}
	\bibitem{3} \url{https://developers.google.com/apps-script/overview}
	\bibitem{2} \url{https://developers.google.com/bigquery/docs/overview}
	\bibitem{5} \url{https://developers.google.com/bigquery/docs/dataset-gsod}
	\bibitem{4} \url{ftp://ftp.ncdc.noaa.gov/pub/data/inventories/ISH-HISTORY.TXT}
	\bibitem{6} \url{https://en.wikipedia.org/wiki/BigQuery}

\end{thebibliography}

\addcontentsline{toc}{section}{Abbildungen}
\renewcommand{\refname}{Abbildungen}
\def\UrlFont{\bfseries}

\begin{appendix}
\section*{Abbildungen}

%\begin{figure}[htbp]
%  \begin{sideways}
%    \begin{minipage}{19cm}
%		\includegraphics[width=0.90\textwidth]{./diagramme/jahresdurchschnitt}
%		\caption{Jahresdurchschnittstemperatur}
%    \end{minipage}
%  \end{sideways}
%  \centering
%\end{figure}


%\begin{sidewaysfigure}[h]
%\centering\includegraphics[width=0.80\textwidth]{./diagramme/monatJahr}
%\caption{Monatsdurchschnittstemperatur über die Jahre}
%\end{sidewaysfigure}

\end{appendix}

\newpage
 


\end{document}
