\section{Einleitung}
\subsection{Idee}
Der Grundgedanke für diesen Beleg bestand darin, einen autonom fahrenden Roboter zu konstruieren und diesen mittels Bilderkennung auf Verkehrszeichen reagieren zu lassen. Da wir hierfür mit einer Webcam arbeiten mussten fiel die Wahl des Roboters auf den EV3 von Lego, da wir eine volle USB-Schnittstelle bei ihm zu Verfügung hatten. Bei der Programmiersprache haben wir uns für Lejos in der Beta 0.81 entschieden, da wir hier rein theoretisch auch die Webcam ansteuern konnten und wir mit Java schon Erfahrungen hatten. \\
Der Roboter selber sollte mit Hilfe von Lichtsensoren des älteren NXT Lego-Models die Linie bzw. den Straßenrand entlangfahren und anhand vonn Helligkeitswerten immer auf der Linie bleiben. Die Webcam sollte die Ganze Zeit Prüfen ob ein Straßenschild am rechten Rand der Straße auftaucht und dieses sollte dann per OpenCV erkannt werden und verarbeitet werden. Anhand der Auswertung sollte der EV3 einen geeigneten Algorithmus auswählen und diesen auch benutzen um auf das Verkehrsschild angemessen zu reagieren.
\subsection{Probleme der Idee}
Hier möchten wir auf Probleme eingehen die schon während der ersten Analyse und Recherche zu unserem Projekt auftraten und wie wir diese versuchten lösen um dieses Projekt erfolgreich abzuschließen. Mehr zu den einzelnen Lösungen und genutzten Technologien finden Sie in den anderen Kapiteln\\
Das größte Problem an sich und welches uns immer wieder Kopfzerbrechen bereitete war die Programmiersprache Lejos an sich. Das lässt sich zum einen auf den Betazustand zurückführen zum anderen aber auch an der Dokumentation dieser Sprache. Wir stellten sehr schnell fest das wir die Webcam nicht mit Lejos nutzen konnten da es hierfür noch keine geeignete Klassen und Methoden gab um diese anzusprechen. Somit war es nicht möglich auf den EV3 allein die Bildverarbeitung und den Fahralgorithmus auszuführen. Nach einiger Webrecherche  kamen wir auf die Idee den Rasberry Pi in das Projekt aufzunehmen. Der Mini-Computer bietet eine volle Linux-Unterstützung, sowie ein komplettes Java und eine OpenCV-Portierung. Der Gedanke dahinter war das die Bildverarbeitung der Rasberry Pi übernimmt und nur entsprechende Signale an den EV3 sendet, damit dieser richtig auf der Schild reagiert. Auch mussten wir die Schildererkennung etwas anpassen damit mir nicht dauernd Signale an den EV3 senden mussten und die Kamera des Pi nicht . Wir haben auf der Fahrbahn eine blaue Markierung gesetzt damit der Rasberry Pi ordentliche Bilder machen kann und diese dann auch ordentlich verarbeitet.\\
Das zweite Problem bestand in der mangelnden Unterstützung der NXT-Sensoren. Hierfür gab es zwar laut der API von Lejos eine Klasse, welche diese Benutzen konnte aber es wurden keine brauchbaren Werte ausgegeben, somit nutzten wir die EV3-Colorsensoren, welche Farbwerte auslesen können. Damit konnten wir die Markierung auf der Straße für die Fotos und die Straße selber differenzieren. \\
Weitere Informationen zur Umsetzung finden Sie in den folgenden Kapiteln

