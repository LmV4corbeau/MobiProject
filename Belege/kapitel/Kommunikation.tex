\section{Kommunikation zwischen Raspberry Pi und Lego EV3}
Nach dem sich abzeichnete, das der Lego EV3 noch nicht mit einer Webcam kommunizieren kann, entschied sich das Team zur Auslagerung der Bildverarbeitung auf einen Raspberry Pi. Für die Kommunikation zwischen Raspberry Pi und EV3 zu bewerkstelligen, wurden drei Wege verfolgt. Zuerst sollte die beiden Geräte über Bluetooth gekoppelt werden, dies schlug durch Probleme mit der Bluez Bibliothek fehl. Durch eine Verbindung des Raspberry Pi mit dem EV3 über USB, funktionierte die Kommunikation.
\subsection{Umsetzung}
Im Server Teil des Projekts, stellt der Raspberry Pi mit dem Befehl \lstinline$ifconfig$ eine TCP/IP-Verbindung zwischen Raspberry Pi und EV3 aufgebaut. Durch den in Java implementierten ProzessBuilder werden die benötigten Bash-Befehle aus dem Java Programm heraus ausgeführt, was zusätzliche Einstellungen vor dem Starten des Servers überflüssig machte. Nach der erfolgreichen Verbindung wird auf dem Raspberry Pi ein Socket geöffnet, über den die beiden Geräte kommunizieren. Auch die Verbindungswiederaufnahme nach der Verbindung nach Absturz des EV3-Programms ist möglich, doch sollte die Hardware Verbindung der beiden Geräte bestehen bleiben.